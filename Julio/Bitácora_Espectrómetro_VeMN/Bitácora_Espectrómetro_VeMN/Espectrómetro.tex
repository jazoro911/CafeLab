\documentclass{article}
\usepackage{tikz}
\usetikzlibrary{angles, quotes}
\usepackage{amssymb}
\usepackage{pgfplots}
\pgfplotsset{width=10cm,compat=1.9}
\usepgfplotslibrary{statistics}
\usepackage[left=2.54cm,right=2.54cm,top=2.54cm,bottom=2.54cm]{geometry}
\usepackage{tasks} %Paquete necesario para la producción de items horizontales para algunos ejercicios de tarea empleando el comando /begin{multicols}}{#}.../end{multicols} para ayudarnos a reducir las páginas
\usepackage{pdflscape} %comando que permite cambiar a orientación horizontal las hojas%
\usepackage{soulutf8} %Paquete para permitir la compilación de acentos usando el comando {\hl{}}
\usepackage{soul} %Paquete para poder subrayar párrafos
\sethlcolor{yellow(munsell)} %comando para definir el color para el subrayado usando el comando \hl
\usepackage[utf8]{inputenc}
\usepackage{booktabs}
\usepackage{siunitx} % para unidades como g, mL, etc.
\usepackage[spanish]{babel}
\usepackage{icomma}
\usepackage{siunitx}
\usepackage{url}
\usepackage[colorlinks=true, urlcolor=blue,  linkcolor=black, citecolor=green]{hyperref}
\usepackage{pdfpages}
\usepackage{blindtext} %Paquete que genera texto ficticio (dummy text) con el comando: \blindtext
\usepackage{cancel} %in the preamble gives you four different modes of striking through: \cancel{text to cancel} draws a diagonal line (slash) through its argument, \bcancel{text to cancel} uses the negative slope (a backslash), \xcancel{text to cancel} draws an X (actually \cancel plus \bcancel), \cancelto{〈value〉}{〈expression〉} draws a diagonal arrow through the 〈expression〉pointing to the 〈value〉 (math-mode only)
\usepackage{csquotes}
\usepackage{afterpage}
\usepackage{parskip} 
\usepackage{float}
\usepackage{enumitem}
\usepackage{multicol}%Paquete que permite la creación aislada de columnas de texto. De esta forma se reduce la cantidad de páginas en nuestro documento
\usepackage{lipsum} %Paquete que genera texto de relleno al igual que \usepackage{blindtext}, se usa con el comando \lipsum[#-#] los #(hashtags) delimitan el número de páginas que deseamos ocupar del paquete lipsum para usarlos como relleno. 
\newenvironment{Figura}
{\par\medskip\noindent\minipage{\linewidth}}
{\endminipage\par\medskip}
\usepackage{caption}
\usepackage[
backend=biber,
sorting=none,
url=true
]{biblatex} %bibliografía
\addbibresource{biblio.bib}
% Establecer el espacio entre las entradas de la bibliografía
\setlength{\bibitemsep}{1\baselineskip} % Puedes ajustar el valor según tus preferencias
\usepackage{amsmath,amsthm,amssymb,amsfonts}
\usepackage{pifont} %Permite la compilación del comando \xmark: es el símbolo de la tachita
\usepackage{mathtools} %permite la compilación de símbolos para matrices
\usepackage{empheq} %Paquete que se relaciona con \usepackage[most]{tcolorbox} para la creación de cuadros/cajas de colores para delimitar los resultados matemáticos o líneas de texto usando la línea de comando: \begin{empheq}[box={\mymath[colback=orange(webcolor)!70,drop lifted shadow]}]{equation*} \end{empheq}
\usepackage[most]{tcolorbox} %Paquete que se relaciona con \usepackage{empheq} para la creación de cuadros/cajas de colores para delimitar los resultados matemáticos o líneas de texto
\renewcommand{\qedsymbol}{$\blacksquare$}
\usetikzlibrary{positioning,decorations.pathreplacing} %Paquete que permite la compilación de llaves para cuadros sinópticos (brace diagram) 
\usepackage{schemata} %The schemata package is designed just for these brace diagrams

\newcommand\AB[2]{\schema{\schemabox{#1}}{\schemabox{#2}}} %comando o paquete necesario para crear cuadros sinópticos

\newtcbox{\mymath}[1][]{%
nobeforeafter, math upper, tcbox raise base,
enhanced, colframe=blue!30!black,
colback=blue!30, boxrule=1pt,
#1} %Comando importante para encasillar los resultados matemáticos en cajas de diferentes colores, se relaciona con los paquetes \usepackage{empheq} y \usepackage[most]{tcolorbox} 

\newenvironment{sysmatrix}[1]
{\left(\begin{array}{@{}#1@{}}}
{\end{array}\right)}
\newcommand{\ro}[1]{%
\xrightarrow{\mathmakebox[\rowidth]{#1}}%
}
\newlength{\rowidth}% row operation width
\AtBeginDocument{\setlength{\rowidth}{3em}}  %Comando importante para laproducción de lineas de operaciones entre matrices, método de Gauss o Gauss Jordan se relaciona con \newenvironment{sysmatrix}

%formato para cambiar el horario a español
\usepackage[spanish]{datetime2}
\DTMsetdatestyle{spanish}

\renewcommand{\today}{\DTMdisplaydate{\the\year}{\the\month}{\the\day }{-1}}
%%%%%%%%%%%%%%%%%%%%%%%%%%%%%%%%%%%%%%%%%

\usepackage{datetime}
\newcommand{\mycurrenttime}{\xxivtime}

%%%%%%%%%%%%%%%%%%%%%%%%%%%%%%%%%%%%%%%%%%%%%%%%%%%%%%%%%%%%%%%%%%%%%%%%%%%%%%%%%%%%%%%%%%%%%%%%%%%%%%%%%%%%%%%%%%%%%%%%%%%%%%%%%%%%%%%%%%%%%%%%%


%%%%%%%%%%%%%Caja de color para el título%%%%%%%%%%%%%%%%%%%%%%%%%%%%%%%%%%%%%%%%%%%%%%%%%%%%%%

\definecolor{myframecolor}{RGB}{85, 100, 19} % Saratoga
\definecolor{myboxcolor}{RGB}{177, 196, 56} % Earls Green

\newtcolorbox{mybox}{
enhanced,
colback=myboxcolor!12, % Color de fondo del cuadro
colframe=myframecolor, % Color del marco del cuadro
arc=0pt,
boxrule=1pt,
borderline west={2mm}{-10mm}{myframecolor}, % Borde en el lado izquierdo
sharp corners=southwest,
width=\linewidth,
before=\par\vspace{\bigskipamount}, % Espacio antes del cuadro
after=\par\vspace{\bigskipamount} % Espacio después del cuadro
}


\newcommand{\euler}{e} %Comando para producir letra e de euler
\newenvironment{remark}{\par\vfill\footnotesize % Vertical white space above the remark and smaller font size

\begin{list}{}{
		\leftmargin=80pt % Indentation on the left
		\rightmargin=60pt}\item\ignorespaces % Indentation on the right
	\makebox[-2.5pt]{\begin{tikzpicture}[overlay]
			\node[draw=Horizon!60,line width=2.5pt,circle,fill=Horizon!25,font=\sffamily\bfseries,inner sep=4pt,outer sep=0pt] at (-19pt,5pt){\textcolor{Horizon}{Nota}};\end{tikzpicture}} % Blue Nota in a circle
	\advance\baselineskip -1pt}{\end{list}\vskip5pt} % Tighter line spacing and white space after remark
	\usepackage{graphicx}
	
	\usepackage{titling}
	
	%colores.tex

%colores que podemos usar en el documento


\usepackage{xcolor}


	%%%%%%%%%%%%%%%%%%%%%%%%%%%%%%%%%%%%%%%%%%%%%%%%%%%%%%%%%%%%Gamas de Azul%%%%%%%%%%%%%%%%%%%%%%%%%%%%%%%%%%%%%%%%%%%%%%%%%%%%%%%%%%%%%%%%%%%%%%%%%%%%%%%%%%%%%%%%%%%%%%%%%%%%%%%%%%%%%%%%%%%%%%%%%%%%%%%%%%%%%%%%%%%%%%%%%%%%%%%%%%%%%%%%%%%%%%%
	
	\definecolor{Cerulean}{RGB}{6, 145, 210} %Azul MercadoPago
	\definecolor{prussianblue}{RGB}{1, 45, 75} %Azul de Prusia
	\definecolor{Blue Sapphire}{RGB}{15, 93, 136}  %Azul Blue Sapphire
	\definecolor{Aguamarina}{rgb}{0.5, 1.0, 0.83} %Azul aguamarina
	\definecolor{trueblue}{rgb}{0.0, 0.45, 0.81} %Azul océano
	\definecolor{Tarawera}{RGB}{6, 48, 70} %Azul fuerte
	\definecolor{palatinateblue}{rgb}{0.15, 0.23, 0.89} %Azul palatinado
	\definecolor{Lochmara}{RGB}{9, 116, 189}  %Azul océano fuerte
	\definecolor{Green vogue}{RGB}{4, 40, 85}  %Azul tenue fuerte
	\definecolor{Horizon}{RGB}{88, 132, 169} % Celeste-cielo fuerte
	\definecolor{blizzardblue}{rgb}{0.67, 0.9, 0.93} %Celeste
	\definecolor{Hippie Blue}{RGB}{92, 148, 179}  %Celeste pastel
	\definecolor{Victoria}{RGB}{67, 68, 140}  %Victoria
	\definecolor{Logan}{RGB}{164, 168, 204}  %Logan
	
	%%%%%%%%%%%%%%%%%%%%%%%%%%%%%%%%%%%%%%%%%%%%%%%%%%%%%%%%%%%%Gamas de Verde%%%%%%%%%%%%%%%%%%%%%%%%%%%%%%%%%%%%%%%%%%%%%%%%%%%%%%%%%%%%%%%%%%%%%%%%%%%%%%%%%%%%%%%%%%%%%%%%%%%%%%%%%%%%%%%%%%%%%%%%%%%%%%%%%%%%%%%%%%%%%%%%%%%%%%%%%%%%%%%%%%%%%%
	
	\definecolor{Surfie Green}{RGB}{12, 131, 123} %Verde Klar
	\definecolor{Bottle Green}{RGB}{8, 52, 28} %Verde alga
	\definecolor{Apple Green}{RGB}{125, 191, 3} %Verde Manzana
	\definecolor{Android Green}{RGB}{156, 196, 61} %Verde Android Green
	\definecolor{Viridian Green}{RGB}{15, 151, 160} % Verde Viridian Green
	\definecolor{tealgreen}{rgb}{0.0, 0.51, 0.5} %Verde Azulado, azul cerceta
	\definecolor{pakistangreen}{rgb}{0.0, 0.4, 0.0} %Verde Pakistán 
	\definecolor{Bluechill}{RGB}{11, 150, 144} %Verde-azulado suave
	\definecolor{Lochinvar}{RGB}{36, 142, 137} %Verde pálido
	\definecolor{Lemon Ginger}{RGB}{170, 164, 40} %Verde lima
	\definecolor{Earls Green}{RGB}{177, 196, 56} %verde claro-manzana
	\definecolor{Saratoga}{RGB}{85, 100, 19}  %Verde-ciénaga  pantano (lodo)
	\definecolor{Deep Sea Green}{RGB}{8, 83, 94} %Verde-azulado tenue oscuro
	\definecolor{tropicalrainforest}{rgb}{0.0, 0.46, 0.37} %Verde Cyan tono oscuro
	\definecolor{brightturquoise}{RGB}{1, 196, 254} %Turquesa Brillante
	\definecolor{Medium Aquamarine}{RGB}{116, 204, 159} %Agua marina
	\definecolor{Elephant}{RGB}{16, 52, 60} %Azul de shell de vi
	
	
	%%%%%%%%%%%%%%%%%%%%%%%%%%%%%%%%%%%%%%%%%%%%%%%%%%%%%%%%%%%%Gamas de Amarillo%%%%%%%%%%%%%%%%%%%%%%%%%%%%%%%%%%%%%%%%%%%%%%%%%%%%%%%%%%%%%%%%%%%%%%%%%%%%%%%%%%%%%%%%%%%%%%%%%%%%%%%%%%%%%%%%%%%%%%%%%%%%%%%%%%%%%%%%%%%%%%%%%%%%%%%%%%%%%%%%%%%
	
	\definecolor{Sun}{RGB}{251, 175, 17} %Amarillo fuerte
	\definecolor{yellow(munsell)}{rgb}{0.94, 0.8, 0.0} %Amarillo fuerte
	\definecolor{Saffron}{RGB}{242, 190, 48} %Amarillo suave
	\definecolor{Mustard}{RGB}{255, 203, 89} %Amarillo mostaza (Mustard)
	
	
	%%%%%%%%%%%%%%%%%%%%%%%%%%%%%%%%%%%%%%%%%%%%%%%%%%%%%%%%%%%%Gamas de Naranja%%%%%%%%%%%%%%%%%%%%%%%%%%%%%%%%%%%%%%%%%%%%%%%%%%%%%%%%%%%%%%%%%%%%%%%%%%%%%%%%%%%%%%%%%%%%%%%%%%%%%%%%%%%%%%%%%%%%%%%%%%%%%%%%%%%%%%%%%%%%%%%%%%%%%%%%%%%%%%%%%%%%
	
	\definecolor{orange(colorwheel)}{rgb}{1.0, 0.5, 0.0} %Naranja fuerte
	\definecolor{carrotorange}{rgb}{0.93, 0.57, 0.13} %Naranja zanahoria
	\definecolor{mandarinaatomica}{rgb}{1.0, 0.6, 0.4} %Naranja claro
	\definecolor{orange(webcolor)}{rgb}{1.0, 0.65, 0.0} %Naranja claro
	\definecolor{tigre}{rgb}{0.88, 0.55, 0.24} %Naranja tigre
	\definecolor{Fiery Orange}{RGB}{180, 92, 22} %Naranja cobre
	
	%%%%%%%%%%%%%%%%%%%%%%%%%%%%%%%%%%%%%%%%%%%%%%%%%%%%%%%%%%%%Gamas de Rojo%%%%%%%%%%%%%%%%%%%%%%%%%%%%%%%%%%%%%%%%%%%%%%%%%%%%%%%%%%%%%%%%%%%%%%%%%%%%%%%%%%%%%%%%%%%%%%%%%%%%%%%%%%%%%%%%%%%%%%%%%%%%%%%%%%%%%%%%%%%%%%%%%%%%%%%%%%%%%%%%%%%%%%%
	
	\definecolor{Milano Red}{RGB}{184, 12, 11} %Rojo fuerte
	\definecolor{Tamarillo}{RGB}{155, 23, 33} %Rojo fuerte, terracota
	\definecolor{Fire Opal}{RGB}{236, 93, 83}  %Rojo Fire Opal
	\definecolor{bittersweet}{rgb}{1.0, 0.44, 0.37} %Rojo sandía
	\definecolor{persianred}{rgb}{0.8, 0.2, 0.2} %Rojo persa-Bermellón
	\definecolor{Cinnabar}{RGB}{225, 71, 53} %Cinabrio
	\definecolor{Mahogany}{RGB}{182, 64, 3} % Cobre-oro
	\definecolor{Ebony Clay}{RGB}{35, 44, 67} %Arcilla de Ébano 
	\definecolor{Tuscany}{RGB}{205, 111, 52} %Marrón
	\definecolor{darkcoral}{RGB}{205, 91, 69} % Color coral tono oscuro
	\definecolor{wildwatermelon}{rgb}{0.99, 0.42, 0.52} %Color rojo-rosa claro
	
	%%%%%%%%%%%%%%%%%%%%%%%%%%%%%%%%%%%%%%%%%%%%%%%%%%%%%%%%%%%%Gamas de Café%%%%%%%%%%%%%%%%%%%%%%%%%%%%%%%%%%%%%%%%%%%%%%%%%%%%%%%%%%%%%%%%%%%%%%%%%%%%%%%%%%%%%%%%%%%%%%%%%%%%%%%%%%%%%%%%%%%%%%%%%%%%%%%%%%%%%%%%%%%%%%%%%%%%%%%%%%%%%%%%%%%%%%%
	
	\definecolor{Metallic Bronze}{RGB}{76, 52, 29} %Metallic Bronze (Café) 
	\definecolor{Pesto}{RGB}{138, 109, 45} %Pesto (Semidorado)
	\definecolor{Coffee Bean}{RGB}{36, 21, 12} %Coffee Bean (café fuerte)
	
	
	%%%%%%%%%%%%%%%%%%%%%%%%%%%%%%%%%%%%%%%%%%%%%%%%%%%%%%%%%%%%Gamas de Morado%%%%%%%%%%%%%%%%%%%%%%%%%%%%%%%%%%%%%%%%%%%%%%%%%%%%%%%%%%%%%%%%%%%%%%%%%%%%%%%%%%%%%%%%%%%%%%%%%%%%%%%%%%%%%%%%%%%%%%%%%%%%%%%%%%%%%%%%%%%%%%%%%%%%%%%%%%%%%%%%%%%%%
	
	\definecolor{Electric Violet}{RGB}{131, 12, 211}%Violeta de NuBank
	\definecolor{antiquefuchsia}{rgb}{0.57, 0.36, 0.51} %Morado
	\definecolor{blue-violet}{rgb}{0.54, 0.17, 0.89} %Morado
	\definecolor{byzantine}{rgb}{0.74, 0.2, 0.64} %Bizancio-púrpura oscuro
	\definecolor{darkmagenta}{rgb}{0.55, 0.0, 0.55} %Magenta oscuro (púrpura oscuro)
	\definecolor{vividviolet}{rgb}{0.62, 0.0, 1.0} %Violeta vívido
	\definecolor{darkviolet}{rgb}{0.58, 0.0, 0.83} %Violeta oscuro	
	\definecolor{plum(traditional)}{rgb}{0.56, 0.27, 0.52} % Color ciruela-violeta	
	\definecolor{deepmagenta}{rgb}{0.8, 0.0, 0.8} %Magenta profundo
	\definecolor{Clairvoyant}{RGB}{48, 4, 60} %Púrpura fuerte
	\definecolor{wisteria}{rgb}{0.79, 0.63, 0.86} %Púrpura claro
	
	%%%%%%%%%%%%%%%%%%%%%%%%%%%%%%%%%%%%%%%%%%%%%%%%%%%%%%%%%%%%Gamas de Rosa%%%%%%%%%%%%%%%%%%%%%%%%%%%%%%%%%%%%%%%%%%%%%%%%%%%%%%%%%%%%%%%%%%%%%%%%%%%%%%%%%%%%%%%%%%%%%%%%%%%%%%%%%%%%%%%%%%%%%%%%%%%%%%%%%%%%%%%%%%%%%%%%%%%%%%%%%%%%%%%%%%%%%%%
	
	\definecolor{ticklemepink}{rgb}{0.99, 0.54, 0.67} %Tono claro de rosa
	\definecolor{tearose(rose)}{rgb}{0.96, 0.76, 0.76} %Rosa claro
	\definecolor{thulianpink}{rgb}{0.87, 0.44, 0.63} %Rosa Thulian
	\definecolor{Cavern Pink}{RGB}{231, 190, 194} %Rosa claro
	\definecolor{Burnt Sienna}{RGB}{236, 119, 88} %Color piel humana
	\definecolor{Hollywood Cerise}{RGB}{236, 6, 141} %Color rosa mexicano fosforescente
	
	%%%%%%%%%%%%%%%%%%%%%%%%%%%%%%%%%%%%%%%%%%%%%%%%%%%%%%%%%%%%Gamas de Blanco%%%%%%%%%%%%%%%%%%%%%%%%%%%%%%%%%%%%%%%%%%%%%%%%%%%%%%%%%%%%%%%%%%%%%%%%%%%%%%%%%%%%%%%%%%%%%%%%%%%%%%%%%%%%%%%%%%%%%%%%%%%%%%%%%%%%%%%%%%%%%%%%%%%%%%%%%%%%%%%%%%%%%%%
	
	\definecolor{Gallery}{RGB}{236, 236, 236} %Blanco-gris
	\definecolor{Iron}{RGB}{227, 227, 228} %Color hierro puro - blanco plateado
	\definecolor{Mercury}{RGB}{228, 228, 228} %Gris claro
	\definecolor{Alto}{RGB}{220, 220, 220} %Gris claro
	\definecolor{bluegray}{rgb}{0.4, 0.6, 0.8} %Gris azulado, lívido
	\definecolor{cinereous}{rgb}{0.6, 0.51, 0.48} % Cinéreo - Gris ceniciento (ceniza)
	\definecolor{coolgray}{rgb}{0.55, 0.57, 0.67} %Gris opaco 
	
	
	%%%%%%%%%%%%%%%%%%%%%%%%%%%%%%%%%%%%%%%%%%%%%%%%%%%%%%%%%%%%Gamas de Gris%%%%%%%%%%%%%%%%%%%%%%%%%%%%%%%%%%%%%%%%%%%%%%%%%%%%%%%%%%%%%%%%%%%%%%%%%%%%%%%%%%%%%%%%%%%%%%%%%%%%%%%%%%%%%%%%%%%%%%%%%%%%%%%%%%%%%%%%%%%%%%%%%%%%%%%%%%%%%%%%%%%%%%%
	
	\definecolor{Nevada}{RGB}{90, 112, 113} %Gris vi id
	\definecolor{Mantle}{RGB}{137, 150, 143} % Gris enfásis vi id
	
	%%%%%%%%%%%%%%%%%%%%%%%%%%%%%%%%%%%%%%%%%%%%%%%%%%%%%%%%%%%%%Gamas de Negro%%%%%%%%%%%%%%%%%%%%%%%%%%%%%%%%%%%%%%%%%%%%%%%%%%%%%%%%%%%%%%%%%%%%%%%%%%%%%%%%%%%%%%%%%%%%%%%%%%%%%%%%%%%%%%%%%%%%%%%%%%%%%%%%%%%%%%%%%%%%%%%%%%%%%%%%%%%%%%%%%%%%%%%%%
	
	\definecolor{onyx}{rgb}{0.06, 0.06, 0.06} %Color Onyx-negro pálido
	
	%%Producir signos de cita textual``Asignatura''
	
	%%Producir signos de cita textual``Asignatura''
	
	\newcommand{\xmark}{\ding{55}} %%%comando que genera la tachita
	
	\newenvironment{MyColorPar}[1]{%
\leavevmode\color{#1}\ignorespaces%
}{%
}%

%%%%%%%%%%%%%%%%%%% Título de la tarea, Nombre de alumno

\title{Espectrómetro. Uso}
\author{\emph{Espectrómetro} $\boldsymbol{\mid}$ Instrucciones}
\date{\today}

%%%%%%%%%%%%%%%%%%% Datos de la Materia

\usepackage{fancyhdr}
\fancypagestyle{plain}{%  the preset of fancyhdr 
\fancyhf{} % clear all header and footer fields
\fancyfoot[R]{\includegraphics[width=2cm]{LogoFCFMBUAP (1).png}}
\fancyfoot[L]{{\bfseries{\thedate{}}} a las {\bfseries{\mycurrenttime{}}} horas{} (GMT-6, H. Puebla de Zaragoza, Pue)}
\fancyhead[L]{Laboratorio de caract. de materiales ({\bfseries{FCFM}}) (BUAP) }
\fancyhead[R]{\theauthor}
}
\makeatletter
\def\@maketitle{%
\newpage
\null
\vskip 1em%
\begin{center}%
	\let \footnote \thanks
	{\LARGE \@title \par}%
	\vskip 1em%
	%{\large \@date}%
\end{center}%
\par
\vskip 1em}
\makeatother

\usepackage{lipsum}  
\usepackage{cmbright}


%%%%%%%%%%%%%%%%%%%%%%%%%%%%%%%%%%%%%%%%%%%%%%%%%%%%%%%%%%%%%%%%%%%%%%%%%%%%%%%%%%%%%%%%%%%%%%%%%%%%%%%%%%%%%%%%%%%%%%%%%%%%%%%%%%%%%%%%%%%%%%%%%%%%%%%%%%%%%%%%%%%%%%%%%%%%%%%%%%%%%%%%%%%%%%%%%%%%%%

\begin{document}



\maketitle



\tableofcontents \vspace{0.5cm}

\section{Pasos} \vspace{0.5cm}

\begin{enumerate}
	\item Prender el espectrómetro \\
	
	\begin{itemize}
		\item[a)] Monitor (pantalla y caja)\\
		\item[b)] Espectro
	\end{itemize} \vspace{0.2cm}
	
	\item Establecer \underline{parámetros}\footnote{Parámetros $\to$ \underline{Absorbancia} (\textcolor{Cinnabar}{\textbf{Data mode}}: \emph{Absorbance}), \( \text{\textcolor{Apple Green}{\textbf{wave length}}} = \left\{ \begin{array}{l}
			start \; \lambda: 200 nm\\
			stop \; \lambda: 410 nm
		\end{array}  \right. \), \textcolor{Cerulean}{\textbf{Data interval}}: Quant, \textcolor{orange(colorwheel)}{\textbf{Lamp Change}}: 340, \textcolor{Sun}{\textbf{Smoothing}} = Medium} de absorción\\
	
	\textbf{Para absorbancia}\\
	
	\item Línea base (con agua)\\
	\item 1 gota de café y dar Run\\
	\item Guardar\\
	
	\textbf{Para transmitancia}\\
	
	\item Establecer \underline{parámetros}\footnote{Parámetros $\to$ \underline{Transmitancia} (\textcolor{Cinnabar}{\textbf{Data mode}}: \emph{\%T}), \( \text{\textcolor{Apple Green}{\textbf{wave length}}} = \left\{ \begin{array}{l}
			start \; \lambda: 390 nm\\
			stop \; \lambda: 750 nm
		\end{array}  \right. \), \textcolor{Cerulean}{\textbf{Data interval}}: Normal, \textcolor{orange(colorwheel)}{\textbf{Lamp Change}}: 340, \textcolor{Sun}{\textbf{Smoothing}} = Low} de transmitancia\\
	\item Línea base (con la gota de café)\\
	\item 5 gotas de café\\
	\item Run
	\item Guardar $\longrightarrow$ \( \left\{ \begin{array}{l}
		1. \text{Expert as ASCII}\\
		2. \text{Data point of SPECTRA}
	\end{array} \right. \)
\end{enumerate}  \vspace{0.5cm}

\section{Para análisis de café de veracruz} \vspace{0.5cm}


Tenemos $\ldots$ \vspace{0.5cm}

\section{\textcolor{Cinnabar}{\textbf{Absorbancia}} \textcolor{Metallic Bronze}{\textbf{café}} - Veracruz ($Ve^3$) }

El subíndice \( Ve^3 \) indica el origen del \textcolor{Metallic Bronze}{\textbf{café}}, correspondiente a la siguiente nomenclatura:\footnote{%
	\(
	\text{Nomenclatura} =
	\left\{
	\begin{array}{lll}
		\text{Veracruz} & \to & Ve \\
		\text{Chiapas} & \to & Ch \\
		\text{Oaxaca} & \to &  Ox \\
		\text{\emph{Mixteca Alta Oaxaqueña}} &  \to & Mi
	\end{array}
	\right.
	\)
}



Proceso inicial: \vspace{0.5cm}

\begin{enumerate}
	\item \emph{Agregar como nuestra línea base, \textcolor{Tarawera}{\bfseries{agua}} para la medición}\\
	\item \textcolor{Cinnabar}{\textbf{Absorbancia}} de \textcolor{Metallic Bronze}{\textbf{café}}  $\boldsymbol{V_{e}}$ \emph{Agregar una gota de \textcolor{Metallic Bronze}{\textbf{café}} al \textcolor{Tarawera}{\bfseries{agua}} para la medición}.  
\end{enumerate} 


 

\section{\textcolor{Apple Green}{\textbf{Transmitancia}} \textcolor{Metallic Bronze}{\textbf{café}} - Veracruz ($V_{e}$)} \vspace{0.5cm}



\begin{enumerate}
	\item \emph{Agregar  \textcolor{Tarawera}{\bfseries{agua}} $+$ una gota de \textcolor{Metallic Bronze}{\textbf{café}} como nuestra línea base, \textcolor{Tarawera}{\bfseries{agua}} para la medición}\\
	\item \textcolor{Apple Green}{\textbf{Transmitancia}} de \textcolor{Metallic Bronze}{\textbf{café}}  $\boldsymbol{V_{e}}$ \emph{Agregar cinco gotas de \textcolor{Metallic Bronze}{\textbf{café}} al \textcolor{Tarawera}{\bfseries{agua}} para la medición}.  
\end{enumerate}  \vspace{0.5cm}







\begin{table}[H]
	\centering
	\caption{Pruebas con diferentes concentraciones de café}
	\label{tabla:pruebas_cafe}
	\begin{tabular}{@{}lccc@{}}
		\toprule
		\textbf{\% de 15 g} & \textbf{20\%} & \textbf{50\%} & \textbf{5\%} \\
		\midrule
		\textbf{Café} & 18 g & 22.5 g & 7.5 g \\
		\textbf{Nombre} & ChMG\_20 & ChMG\_50 & ChMG\_05 \\
		\textbf{Agua ({H2O})} & \multicolumn{3}{c}{200 mL} \\
		\textbf{Tiempo} & \multicolumn{3}{c}{4 min (se consideraron 5 min)} \\
		\textbf{Temperatura} & \multicolumn{3}{c}{92\,\textdegree C} \\
		\bottomrule
	\end{tabular}
\end{table} \newpage

\section*{Preparación de café - Muestras Veracruz}

\begin{enumerate}
	\item Se prepararan 4 muestras con distintas cantidades de café (no importa el orden):
	\begin{itemize}
		\item 1$^{\text{ra}}$ muestra: 15 g
		\item 2$^{\text{da}}$ muestra: 7.5 g
		\item 3$^{\text{ra}}$ muestra: 18 g
		\item 4$^{\text{ta}}$ muestra: 22.5 g
	\end{itemize}
	
	\item Todas las muestras preparadas con la misma cantidad de agua: \textbf{200 mL}.
	
	\item La temperatura máxima del agua debe ser de \textbf{93\,$^\circ$C}.
	
	\item Dejar reposar cada muestra durante \textbf{5 minutos} en la prensa antes de llevar a espectrómetro y continuar con el análisis.
	
	\item El agua se calentó hasta que comenzó a hervir.
	
	\begin{itemize}
		\item \textbf{Spark value} (software para análisis)
		\item \textbf{Termómetro de spark value}
		\item \textbf{GUI (graphic user interface) (iPad)} para control y visualización de datos.
	\end{itemize}
\end{enumerate} \vfill

\vspace{0.5cm}

\section{Prueba de laboratorio — Café Veracruz (Ve)} \vspace{0.5cm}

\subsection*{Pesaje de las muestras}
\begin{itemize}
	\item $1^\text{ra}$ Muestra: 15.0 g $\Rightarrow$ \textbf{real}: 15.1013 g
	\item $2^\text{da}$ Muestra: 7.5 g $\Rightarrow$ \textbf{real}: 7.5136 g
	\item $3^\text{ra}$ Muestra: 18.0 g $\Rightarrow$ \textbf{real}: 18.0874 g
	\item $4^\text{ta}$ Muestra: 22.5 g $\Rightarrow$ \textbf{real}: 22.5737 g
\end{itemize} \vspace{0.5cm}

La muestra de agua utilizada en cada extracción alcanza temperaturas entre 90 y 95.3°C. Todas las preparaciones se hicieron con \textbf{200 ml} de agua, y se purgó previamente la prensa francesa.

---

\subsection{Preparación con muestra de 7.5 g}

\begin{itemize}
	\item Se calienta agua para purgar la prensa francesa (90–94°C).
	\item Se colocan \textbf{200 ml} de agua para la extracción.
	\item Se añade café de muestra (7.5136 g) a la prensa.
	\item Se espera 1 minuto para que el agua (a 95°C) enfríe ligeramente.
	\item Se realiza una preinfusión de 30 segundos para liberar CO\textsubscript{2}.
	\item Se vierte el resto del agua, se remueve y se deja reposar por 5 minutos.
	\item A las 12:33 del 19/06/2025, se extrae y cuela el café.
\end{itemize}

La temperatura del agua al verterse fue de \textbf{93°C}. \underline{(Investigar el principio de enfriamiento de Newton)}.

Posteriormente, la muestra fue llevada al espectrómetro para medir:
\begin{itemize}
	\item \textcolor{Cinnabar}{\textbf{Absorbancia}}
	\item \textcolor{Apple Green}{\textbf{Transmitancia}}
\end{itemize}

---

\subsection{Preparación con muestra de 18 g}

\begin{itemize}
	\item Agua para purgar la prensa: 91–92°C.
	\item Se colocan 200 ml para preparar café con 18.0874 g.
\end{itemize}

\subsubsection*{Ratio de extracción}
\[
\frac{200 \, \text{ml}}{18 \, \text{g}} = 11.\overline{11} \quad \Rightarrow \quad \textbf{Ratio: } 1:11.\overline{11}
\]

\begin{itemize}
	\item Se purga la prensa.
	\item Se hace preinfusión de 30 seg.
	\item Se vierte el resto del agua y se deja reposar 5 minutos.
	\item A las 13:43 del 19/06/2025 se extrae el café.
\end{itemize}

El café resultante fue notablemente más concentrado, posiblemente debido al ratio más bajo que el estándar (\textit{1:14} a \textit{1:16}).

---

\subsection{Preparación con muestra de 22.5 g}

\begin{itemize}
	\item Se calienta agua para purgar la prensa (92–93°C).
	\item Se colocan 200 ml para preparar café con 22.5737 g.
	\item Se hace preinfusión de 30 seg para liberar CO\textsubscript{2}.
	\item Se vierte el resto del agua y se deja reposar por 5 minutos.
	\item Infusión inicia a las 14:09 y finaliza a las 14:14 del 19/06/2025.
\end{itemize}

\subsubsection*{Ratio de extracción}
\[
\frac{200 \, \text{ml}}{22.5 \, \text{g}} = 8.88 \quad \Rightarrow \quad \textbf{Ratio: } 1:8.88
\]





\printbibliography[heading=bibintoc]




\end{document}
